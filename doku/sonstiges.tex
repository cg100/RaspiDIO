\section{Sonstiges}
\subsection{Automatischer Programmstart}
Damit nach dem Booten des Linux Betriebssystems RaspiDIO sofort automatisch gestartet wird, wurde ein Startup-Skript angelegt.
\begin{lstlisting}[label=startup,caption=/etc/init.d/raspidio]
#! /bin/sh
# /etc/init.d/raspidio

touch /var/lock/raspidio

case "$1" in
start)
echo "Starting RaspiDIO..."
/home/pi/PiRadio2> /dev/null &
;;
stop)
echo "Killing RaspiDIO ..."
killall PiRadio2
;;
*)
echo "Usage: /etc/init.d/raspidio start|stop}"
exit 1
;;
esac
exit 0
\end{lstlisting}
Dieses muss dann mit \\
"`chmod 755 /etc/init.d/raspidio"'\\
ausführbar gemacht werden und mittels \\
"`update-rc.d raspidio defaults"' \\
in die Liste der beim Systemstart ausgeführten Programme eingetragen werden..
\subsection{SD-Karte klonen}
Die SD-Karte lässt sich unter Mac OS X mit dem Kommandozeilenbefehl dd klonen und Wiederherstellen.
\begin{lstlisting}[label=clone,caption=Klonen und Wiederherstellen der SD-Karte]
//Inhalt von SD-Karte auf dem Desktop speichern
sudo dd if=/dev/disk1 of=~/Desktop/raspberrypi.dmg

//Image auf SD-Karte wiederherstellen
diskutil unmountDisk /dev/disk1
sudo newfs_msdos -F 16 /dev/disk1
sudo dd if=~/Desktop/raspberrypi.dmg of=/dev/disk1
\end{lstlisting}