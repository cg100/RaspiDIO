\section{Infrarot}
Zur Benutzereingabe steht auch eine Infrarot-Fernbedienung sowie ein Infrarot-Empfänger zur Verfügung. 

\subsection{Hardware}
Der Infrarot-Empfänger TSOPXXXX wird auch an die GPIOs des Raspberrys angeschlossen. 
\newline
Es wird nur ein Eingangspin benötigt. Wir haben den Empfänger an GPIO Pin 17 angeschlossen. Außerdem benötigt der Empfänger eine 3,3 Volt Stromversorgung und einen Masseanschluss. 

\subsection{Software}
Damit das von der Fernbedienung empfangene Signal zuverlässig dekodiert werden kann setzen wir den Treiber LIRC ein.
\subsubsection{LIRC}
LIRC ist die Abkürzung für Linux Infrared Remote Control. Mit LIRC kann man Signale, die per Infrarot empfangen werden in Programmbefehle übersetzen.
\newline
Als erstes muss lirc auf dem Raspberry installiert werden. Dafür gibt es den Befehl:
\newline
"`sudo apt-get install lirc"'
\newline
Ist der Treiber installiert, muss er geladen werden. Dabei muss der GPIO Pin angegeben werden, an dem der Infrarot-Empfänger angeschlossen ist.
\newline
"`sudo modprobe lirc\_rpi gpio\_in\_pin=17"'
\newline
Damit der Treiber beim booten automatisch geladen wird, muss er in der Datei "`/etc/modules"' eingetragen werden.
\newline
\newline
Man könnte den Empfänger jetzt schon testen, da es aber noch keine fernbedienungsspezifische Konfigurationsdatei gibt, werden nur wirre Zeichen angezeigt.
Deshalb legen wir als nächstes die Konfigurationsdatei für unsere Fernbedienung vom Typ "`Apple A1156"' an.

\begin{lstlisting}[label=infra_config,caption=Fernbedienung Konfiguration]
# this config file was automatically generated
# using lirc-0.8.1(iguanaIR) on Fri Mar 30 19:20:40 2007
#
# contributed by Matthias Urlichs <matthias|urlichs.de>
#
# brand:                       Apple
# model no. of remote control: A1156
# devices being controlled by this remote: new (late 2006) MacBook
#
# This config files are for non-Apple receivers only.
# Use the lircd.conf.macmini file when you are using the Apple receiver.
# 

begin remote
  name  Apple_A1156
  bits            8
  flags SPACE_ENC
  eps            30
  aeps          100

  header       9065  4484
  one           574  1668
  zero          574   547
  ptrail        567
  repeat       9031  2242
  pre_data_bits   16
  pre_data       0x77E1
  post_data_bits  8
  post_data      0xC5
  gap          37600
  toggle_bit      0
  ignore_mask 0x80ff

      begin codes
          play                     0x20
          plus                     0xD0
          ffwd                     0xE0
          rev                      0x10
          minus                    0xB0
          menu                     0x40
      end codes

end remote
\end{lstlisting}

Sie wird in der Datei "`/etc/lirc/lircd.conf"' gespeichert.
\newline
Nun kann die Fernbedienung und der Empfänger mit dem Programm "`irw"' getestet werden. Es werden jetzt die Befehle angezeigt, die in der Konfigurationsdatei definiert worden sind.

Konfigurationsdateien für viele Fernbedienungen gibt es unter
http://lirc.sourceforge.net/remotes/
\newline
\newline
LIRC stellt auch eine Bibliothek zur Einbindung in eigene Programme bereit. Dazu muss die "`lirc\_client.h"' eingebunden werden.
\newline
\newline
Nun kann nach der Initialisierung von LIRC der aktuelle Befehl mit "`lirc\_nextcode()"' empfangen werden.
\begin{lstlisting}[label=infra_quellcode,caption=LIRC Programmierung]
//Initialisierung
struct lirc_config *config;
lirc_init("irexec",1);
lirc_readconfig(NULL,&config,NULL)

char *code;
char *c;

//Befehl holen
while(lirc_nextcode(&code)==0)
{
	lirc_code2char(config,code,&c))
	system(c);
	free(code);
 }
 lirc_freeconfig(config);
\end{lstlisting}